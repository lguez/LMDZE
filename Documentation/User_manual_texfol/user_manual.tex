%; whizzy subsubsection   
\documentclass[a4paper,english,french]{article}

\usepackage[utf8]{inputenc}

\usepackage{amsmath}
\usepackage{amsfonts}
\usepackage[T1]{fontenc}
\usepackage{lmodern}
\usepackage{algorithmic}
\usepackage{babel}
\usepackage{color}
\usepackage{graphicx}
\usepackage{mathabx}
\usepackage{mathrsfs}
\usepackage{variations}
\usepackage[all]{xy}
\usepackage[np]{numprint}

\usepackage{hyperref}

\hypersetup{pdftitle={Installation et utilisation comme boîte noire},
  pdfauthor={Lionel Guez}}

\newcommand{\ud}{\mathrm{d}}
\newcommand{\uD}{\mathrm{D}}

\newcommand{\Eng}[1]{\textit{\foreignlanguage{english}{#1}}}
\newcommand{\pseudov}[1]{\overset{\curvearrowbotright}{#1}}

\DeclareMathOperator{\diverg}{div}
\DeclareMathOperator{\tgh}{th}
\DeclareMathOperator{\Arcsin}{Arcsin}

\renewcommand{\algorithmicdo}{\textbf{faire}}
\renewcommand{\algorithmicend}{\textbf{fin}}
\renewcommand{\algorithmicfor}{\textbf{pour}}

\graphicspath{{Graphiques/}}

\author{Lionel GUEZ}
\title{Manuel pour LMDZE -- Installation et utilisation comme boîte noire}

\begin{document}

\maketitle
\tableofcontents

\section{Divers}

La résolution spatiale est définie par les variables IIM, JJM et LLM
(dans \verb+dimensions_nml+).

Cf. figure (\ref{fig:inout_LMDZE}).
\begin{figure}[htbp]
  \centering
  \includegraphics{inout_LMDZE}
  \caption{Fichiers d'entrée et de sortie.}
  \label{fig:inout_LMDZE}
\end{figure}
Les fichiers \verb+ECDYN.nc+ et \verb+ECPHY.nc+ contiennent des
données de la réanalyse ERA interim de l'ECMWF, pour un jour
particulier, avec une résolution de $1^\circ$. Il peut être
intéressant de refaire ces fichiers pour la date initiale choisie.

Le fichier \verb+amipbc_sic_1x1.nc+ donne la fraction de la surface de
la mer couverte par de la glace. Les fichiers \verb+amipbc_sic_1x1.nc+
et \verb+amipbc_sst_1x1.nc+ viennent du projet
\href{http://www-pcmdi.llnl.gov/projects/amip/AMIP2EXPDSN/BCS/bcsintro.php}{AMIP
  II}.

\verb+Rugos.nc+ contient la contribution saisonnière de la longueur de
rugosité (excluant la contribution des variations de relief). Ces
données viennent de la NASA. Le fichier a probablement été récupéré
par Laurent Li, mais il ne se souvient rien de plus. \verb+Albedo.nc+
contient probablement des données de Yann Polcher.

Dans le fichier \verb+landiceref.nc+, la variable NetCDF
\verb+landice+ prend des valeurs différentes de 0 et 1, entre
\np{0,125} et \np{0,875}, en un petit nombre de points (163 points),
sur des côtes. Cf. figure (\ref{fig:landiceref}).
\begin{figure}[htbp]
  \centering
  \includegraphics[width=\textwidth]{landiceref}
  \caption{Valeurs non nulles de landice dans \texttt{landiceref.nc},
    près du pôle nord. Projection Lambert azimutale équivalente.}
  \label{fig:landiceref}
\end{figure}

Cf. la \href{file://../namelists.ods}{liste des paramètres d'entrée}.
Avant de lancer l'exécution de \verb+gcm+, il faut en particulier
choisir \verb+nday+ (par exemple 1 ou 30). Je peux aussi modifier
\verb+periodav+, qui contrôle la fréquence d'écriture dans le fichier
\verb+dynzon.nc+.

Phu Le Van conseille de diminuer à 3600 les valeurs de \verb+tetagdiv+
et \verb+tetagrot+.

\verb+grossismx+ et \verb+grossismy+ doivent être inférieurs à 4 pour
que le modèle converge.

La résolution horizontale standard est de 96 longitudes $\times$ 72
latitudes. (Un avantage annexe est que la grille des latitudes tombe
alors sur des valeurs rondes.) LMDZ ne converge plus lorsque la
résolution devient trop fine. Cf. aussi la documentation officielle de
LMDZ4 sur le
\href{http://www.lmd.jussieu.fr/~lmdz/LMDZ4/choisir_une_resolution.html}{choix
  d'une résolution horizontale}.

Pour guider, il  faut interpoler à l'avance horizontalement sur la
grille du modèle. Cf. script \verb+interp_from_era.sh+.

Pour un traceur, il n'est pas obligatoire de choisir une condition à
la surface, on peut éventuellement se contenter d'une distribution
initiale et de transporter. Pour ajouter un traceur, ajouter une ligne
dans traceur.def.

Dans les fichiers d'historiques, la variable NetCDF bils contient la
somme des variables NetCDF flat, sols, soll, sens. Tous ces flux sont
positifs vers le bas.

La variable NetCDF ve contient l'intégrale verticale sur toute la
hauteur de l'atmosphère de la composante méridienne de la densité de
flux d'advection d'énergie statique sèche. En notant $h_d$ l'énergie
statique sèche par unité de masse :
\begin{equation*}
  h_d = c_{pd} T + \Phi
\end{equation*}
$\rho v h_d$ est la composante méridienne de la densité de flux
d'advection d'énergie statique et on a :
\begin{equation*}
  \mathrm{"ve"} = \int v h_d \frac{\ud p}{g} = \int \rho v h_d\ \ud z
\end{equation*}
ve est calculé par la procédure transp. ve est un flux d'énergie dans
la direction méridienne par unité de longueur dans la direction
zonale. Le nom de la variable correspondante dans CMIP 6 est intvadse.
\begin{equation*}
  a \cos \phi \int \mathrm{"ve"} \ud \lambda
  = \iint \rho h_d \mathbf{U} \cdot \pseudov{\ud^2 S}
\end{equation*}
est le flux d'advection d'énergie statique sèche à travers la latitude
$\phi$ (ou encore à travers le cône de sommet le centre de la Terre,
d'axe l'axe de rotation de la Terre, et de latitude $\phi$).
\begin{equation*}
  \mathrm{"vq"} = \int v q \rho\ \ud z
\end{equation*}

\section{Paramètres liés au pas de temps}

Les paramètres à choisir au moment de l'exécution et liés au pas de
temps sont \verb+day_step+, \verb+iphysiq+, \verb+iperiod+ et
\verb+iconser+. Il faut en particulier penser à ajuster ces paramètres
lorsque l'on a changé la résolution spatiale. Le principe de base est
que le pas de temps doit être suffisamment petit pour que le modèle
soit stable. Il doit être d'autant plus petit que la résolution
spatiale horizontale est fine. Si les milieux de couches ne montent
pas trop haut dans la stratosphère alors la résolution verticale n'a
pas à être prise en compte. La référence est un pas de temps maximal
de 6 mn environ pour une grille de 64 longitudes par 50 latitudes.
Lorsqu'on multiplie par un même facteur le nombre de longitudes et de
latitudes, on divise par ce facteur le pas de temps maximal.

\verb+day_step+ est le nombre de pas de temps par jour. \verb+iphysiq+
est le nombre de pas de temps entre deux appels de la physique.
\verb+iperiod+ est le nombre de pas de temps entre deux pas Matsuno.
\verb+iconser+ est le nombre de pas de temps entre deux écritures des
variables de contrôle. Ces quatre paramètres doivent recevoir des
valeurs entières. Apparemment, on laisse toujours \verb+iperiod+ à 5
(quelle que soit la résolution).

On a en outre la contrainte que \verb+day_step+ soit un multiple de
\verb+iperiod+, c'est-à-dire normalement un multiple de 5.

Enfin, on a la contrainte suivante sur \verb+iphysiq+. Dans fisrtilp,
on trouve :
\begin{verbatim}
IF (abs(dtime / real(ninter) - 360.) > 0.001) THEN
   PRINT *, "fisrtilp : ce n'est pas prévu, voir Z. X. Li", dtime
   PRINT *, 'Je préfère un sous-intervalle de 6 minutes.'
END IF
\end{verbatim}
Donc, comme \verb+ninter+ vaut 5, il faut que \verb+dtime+, le pas de
temps de la physique, vaille 1800 s à 5.10$^{-3}$ s près. Notons $d$
le pas de temps de la dynamique, en mn. Il faut donc que
$\mathtt{iphysiq} \times d = 30$.

En résumé, en notant $d_M$ le pas de temps maximal imposé par la
résolution, nous avons le système de contraintes suivant :
\begin{displaymath}
  \left\{
    \begin{array}{l}
      \mathtt{iphysiq} \times d = 30 \\
      d \le d_M \\
      \mathtt{day\_step} \times d = 1440 \\
      \mathtt{day\_step} \equiv 0[5]
    \end{array}
  \right.
\end{displaymath}
D'où :
\begin{displaymath}
  \left\{
    \begin{array}{l}
      \mathtt{iphysiq} \ge = 30 / d_M \\
      48 \times \mathtt{iphysiq} \equiv 0[5]
    \end{array}
  \right.
\end{displaymath}
C'est-à-dire :
\begin{displaymath}
  \left\{
    \begin{array}{l}
      \mathtt{iphysiq} \ge = 30 / d_M \\
      \mathtt{iphysiq} \equiv 0[5]
    \end{array}
  \right.
\end{displaymath}
On peut donc adopter l'algorithme suivant, pour un maillage sans zoom :
\begin{verbatim}
entrer(iim, jjm)
n := ceiling(max(iim / 64., jjm / 50.))
iphysiq := 5 * n
day_step := 240 * n
\end{verbatim}
\begin{align*}
  & \min_{|\phi| \le 60^\circ} (\cos \phi \Delta \lambda)
  = \frac{\Delta \lambda}{2} = \frac{180^\circ}{\mathtt{iim}} \\
  & \min \Delta \phi = \Delta \phi = \frac{180^\circ}{\mathtt{jjm}}
\end{align*}
Tant que jjm $\le$ iim et que le filtre commence à $60^\circ$, le pas de
temps est déterminé par iim. Si iim = jjm alors $\delta x = \delta y$
à $60^\circ$ de latitude.

Pour un maillage avec zoom, la contrainte de stabilité est donnée par
$\min \Delta \phi$ et :
\begin{equation*}
  \min_{|\phi| \le 60^\circ} (\cos \phi \Delta \lambda)
  = \frac{\min \Delta \lambda}{2}
\end{equation*}
Il suffit donc de choisir le pas de temps adéquat pour un maillage sans
zoom de résolution $\min \Delta \phi$ et $\min \Delta \lambda$. $\min
\Delta \phi$ et $\min \Delta \lambda$ sont écrits sur la sortie
standard.

Si on guide alors on lit les champs de guidage quatre fois par jour
(toutes les 6 h) et on guide tous les iperiod pas de temps. Non
seulement 6 h doit tomber sur un pas de temps, c'est-à-dire que
\verb+day_step+ doit être un multiple de 4 mais 6 h doit être un pas
de temps où l'on guide, c'est-à-dire que \verb+day_step/4+ doit être
un multiple de iperiod. La condition nécessaire et suffisante est donc
que \verb+day_step+ soit un multiple de $4 \times$ iperiod.

\section{Ressources nécessaires}

Espace disque par utilisateur :
\begin{itemize}
\item fichiers sources LMDZE (sans NetCDF) 12 MiB ;
\item fichiers objets, exécutables (sans NetCDF) (résolution
  $16 \times 12 \times 11$) 38 MiB ;
\item fichiers sortie \verb+ce0l+ 9 MiB ;
\item fichiers sortie \verb+gcm+ (principalement \verb+histday.nc+) 26
  MiB.
\end{itemize}
Par ailleurs, on peut centraliser :
\begin{itemize}
\item fichiers NetCDF entrée \verb+ce0l+ 40 MiB ;
\item ensemble NetCDF (bibliothèque, utilitaires, manuels, modules,
  etc.) 3 MiB.
\end{itemize}

Mémoire principale (résolution $32 \times 24 \times 9$) : 30 MiB.

Durées. Temps de compilation avec \verb+g95 -O3+ : 5 mn. Temps
d'exécution de \verb+gcm+ avec \verb+g95 -O3+ (résolution $32 \times
24 \times 9$) : environ 10 mn par mois simulé.

Il est possible d'économiser de la mémoire vive en changeant la valeur
de kdlon dans \verb+phylmd/raddim.f90+.  Pour cela, consulter le
tableau (\ref{tab:kdlon}), les
\href{http://www.lmd.jussieu.fr/~lmdz/LMDZ4/choisir_une_resolution.html}{conseils}
et le programme
\href{file:///user/guez/Documents/Informatique_fonctionnement/Tests/LMDZ/choixdim.f}{\texttt{choixdim}}.
\begin{table}[htbp]
  \centering
  \begin{tabular}{lll}
    \texttt{iim} & \texttt{jjm} & \texttt{kdlon} \\
    \hline
    32 & 24 & 41 \\
    48 & 32 & 149 \\
    72 & 46 & 1621 \\
    96 & 72 & 487 \\
    160 & 98 & 78 \\
    192 & 43 & 10
  \end{tabular}
  \caption{Choix de \texttt{kdlon} dans \texttt{phylmd/raddim.f90}.}
  \label{tab:kdlon}
\end{table}

\end{document}

%%% Local Variables:
%%% mode: latex
%%% TeX-master: t
%%% End:
