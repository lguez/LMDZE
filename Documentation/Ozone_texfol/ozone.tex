\documentclass[a4paper,english,french]{article}

\usepackage[utf8]{inputenc}

\usepackage{amsmath}
\usepackage[T1]{fontenc}
\usepackage{lmodern}
\usepackage{algorithmic}
\usepackage{babel}
\usepackage{graphicx}
\usepackage{mathabx}
\usepackage[all]{xy}
\usepackage[np]{numprint}
\usepackage{hyperref}

\hypersetup{pdftitle={Traitement de l'ozone}, pdfauthor={Lionel Guez},
  hypertexnames=false, pdfstartview=FitBH}

\newcommand{\ud}{\mathrm{d}}
\newcommand{\uD}{\mathrm{D}}
\newcommand{\Eng}[1]{\textit{\foreignlanguage{english}{#1}}}
\newcommand{\pseudov}[1]{\overset{\curvearrowbotright}{#1}}

\renewcommand{\algorithmicdo}{\textbf{faire}}
\renewcommand{\algorithmicend}{\textbf{fin}}
\renewcommand{\algorithmicfor}{\textbf{pour}}

\DeclareMathOperator{\diverg}{div}
\graphicspath{{Graphiques/}}

\author{Lionel GUEZ}
\title{Traitement de l'ozone}

                                %---------------

\begin{document}

\maketitle
\tableofcontents
\listoffigures
\newpage

Résumé. On part d'une paramétrisation de la chimie stratosphérique de
l'ozone, élaborée par Daniel Cariolle et Hubert Teyssèdre (version
2.8). La paramétrisation calcule en chaque point une production
chimique d'ozone. On suit l'ozone ainsi produit comme un traceur. Les
coefficients chimiques sont remaillés pour LMDZ, en latitude et en
temps, dans \verb+ce0l+. Le remaillage en temps est une interpolation
linéaire. Le remaillage vertical est fait une fois par jour dans
\verb+gcm+. On néglige donc l'effet sur le remaillage de la variation
temporelle de pression à la surface pendant une journée.  Le champ
initial d'ozone spécifié dans \verb+ce0l+ est le champ de référence de
D. Cariolle (un des coefficients). Pour les passages de fraction
molaire d'ozone à fraction massique, on néglige la présence de vapeur
d'eau dans l'air.  L'évolution de la fraction massique d'ozone pendant
un pas de temps de la physique est calculée dans \verb+phytrac+ : les
contributions du transport et de la chimie sont intégrées
séparément. La température, la masse volumique de l'air et la
direction du Soleil sont considérées comme constantes pendant un pas
de temps de la physique. Un des coefficients de D. Cariolle est la
densité-colonne d'ozone au dessus d'un point donné. Cette
densité-colonne est calculée dans l'approximation ``plans
parallèles''. La production chimique est intégrée pendant un pas de
temps de la physique par la méthode du point milieu. Une fois par jour
de simulation, dans \verb+gcm+, une procédure lit dans un fichier tous
les coefficients chimiques au bon jour, remaillés en latitude pour
LMDZ.

Cf. Cariolle (1986 761),
\href{file:/user/guez/Documents/Commentaires_lectures/Autres_publications/-1991/Cariolle_1986_761.texfol/Cariolle_1986_761.dvi}{commentaires
  sur Cariolle (1986 761)}, texte des courriels (François L., 29/11/6
; D. Cariolle, 29/11/6 ; François L., 12/1/7),
\href{file:///user/guez/Documents/Informatique_fonctionnement/O3_Cariolle.txt}{fiche}
de D. Cariolle, \verb+coefoz_v2_8.nc+, Cariolle (2007 772).  Attention
: les symboles ne désignent apparemment pas les mêmes grandeurs dans
Cariolle (1986 761) et dans la paramétrisation \verb+2.8+.  $r$ passe
d'une concentration massique à une fraction molaire.  $P$ et $L$
passent de la dimension masse$^{-1}$ temps$^{-1}$ à la dimension
temps$^{-1}$.

Frédéric H. et Phu L. V. indiquent qu'il est nécessaire d'utiliser le
schéma de transport numéro 10 (choix dans \verb+traceur.def+).

Nous notons avec un indice ``Mob'' les valeurs des coefficients de D.
Cariolle et H. Teyssèdre, par référence au modèle Mobidic.

Est-il utile de faire subir à l'ozone les traitements de
\verb+phytrac+ : effets de la convection, des ``thermiques'', de la
couche limite, du lessivage ? Frédéric H. pense que oui. Cf. l'extrait
de sa
\href{file:///user/guez/Documents/Hourdin_Amma_excerpt.pdf}{présentation
  à une réunion AMMA}. Il compare le champ troposphérique d'ozone
obtenu par un modèle incluant la chimie (LMDZ-Inca ou Mocage) et le
champ troposhérique d'un traceur stratosphérique idéalisé. On voit que
la distribution troposphérique d'ozone dans un modèle incluant la
chimie doit être influencée en grande partie par l'ozone d'origine
stratosphérique. Et la distribution troposphérique s'explique bien par
les phénomènes de convection. Donc on obtiendrait une distribution
troposhérique d'ozone non dénuée d'intérêt en appliquant les
phénomènes troposphériques de transport sous-maille à l'ozone de Cariolle.

\section{La paramétrisation de la chimie}

Le taux de production photochimique par unité de volume (ou de masse
d'air) est le nombre de molécules créées par unité de volume (ou de
masse d'air) et par unité de temps. On peut aussi définir un taux de
destruction chimique d'une espèce, de dimension temps$^{-1}$, sachant
que le nombre de molécules de cette espèce détruites par unité de
volume et par unité de temps est proportionnel à la densité de
l'espèce. Comment $P$ est-il donc défini dans la paramétrisation de
2005 ?  En divisant par $N$, la densité totale de la phase gazeuse ?
Quelle définition pour $L$ ? Par ailleurs, quelle grandeur la
paramétrisation de 2005 permet-elle de calculer ? Certainement pas la
dérivée partielle comme noté dans l'équation (5) de Cariolle (1986
761), car la dérivée partielle doit dépendre du mouvement de
l'atmosphère. Cf.
\href{file:/user/guez/Documents/Commentaires_lectures/Autres_publications/-1991/Cariolle_1986_761.texfol/Cariolle_1986_761.dvi}{commentaires
  sur Cariolle (1986 761)}. La paramétrisation donne-t-elle la
dérivée lagrangienne de la fraction molaire, comme noté dans la
\href{file:///user/guez/Documents/Informatique_fonctionnement/O3_Cariolle.txt}{fiche}
de D. Cariolle ?  Notons :
\begin{description}
\item[$r$] : fraction molaire d'ozone
\item[$q$] : fraction massique d'ozone
\item[$s_{O_3}$] : masse nette d'ozone créée par la chimie par unité de
  volume et par unité de temps
\item[$S$] : nombre net de molécules d'ozone créées par la chimie par
  unité de volume et par unité de temps
\item[$N$] : densité totale de la phase gazeuse
\item[$\rho$] : masse volumique totale de la phase gazeuse
\end{description}
On a :
\begin{equation}
  \label{eq:o3}
  \left\{
    \begin{array}{l}
      \partial_t \rho_{\mathrm{O}_3} + \diverg(\rho_{\mathrm{O}_3} \vec v)
      = s_{O_3} \\
      \partial_t \rho + \diverg(\rho \vec v) = 0
    \end{array}
  \right.
\end{equation}
Remarquer que l'équation bilan sur le nombre total de molécules en
phase gazeuse est plus compliquée que celle sur la masse totale : le
second membre n'est a priori pas nul. D'après l'équation (\ref{eq:o3})
:
\begin{equation}
  \label{eq:dqdt}
  \frac{\uD q}{\uD t} = s_{O_3} / \rho
\end{equation}
D'où :
\begin{displaymath}
  \frac{\uD r}{\uD t}
  =
  \frac{1}{\mu_\mathrm{moy}} \frac{\uD \mu_\mathrm{moy}}{\uD t} r + S / N
\end{displaymath}
Donc il serait bizarre que la paramétrisation donne $\uD r/\uD t$. Je
vais supposer qu'elle donne $S/N$. $P$ et $L$ sont alors les taux de
production et de destruction chimique par unité de volume divisés par
$N$. Je note : $P_\mathrm{net} := S / N$, taux de production chimique
d'ozone, par molécule d'air.

Désignons par $R_\mathrm{het}$ le taux de production chimique
hétérogène par molécule d'ozone, interpolé journalièrement, remaillé
horizontalement et verticalement pour LMDZ. Notons $\phi$ la latitude
et $\lambda$ la longitude. Notons $d(t)$ le jour contenant une date
$t$, $s$ la coordonnée verticale hybride et $\alpha$ l'angle entre la
direction du Soleil et la verticale (\Eng{solar zenith angle?}). Nous
pouvons définir un taux de destruction par chimie hétérogène corrigé :
\begin{multline*}
  R_\mathrm{het,corr}(\lambda, \phi, s, t)
  =
  R_\mathrm{het}(\lambda, \phi, s, d(t))
  1_{T(\lambda, \phi, s, t) > 195\ \mathrm{K}}
  1_{\phi \notin [- 45^\circ, 45^\circ]} \\
  \times 1_{\alpha(\lambda, \phi, t) < 87^\circ}
  \left(\frac{\mathrm{Clx}}{\np{3,8} \cdot 10^{-9}}\right)^2
\end{multline*}
(La définition de Clx n'est pas claire pour moi.) Clx est supposé être
une constante. On calcule alors :
\begin{multline*}
  P_\mathrm{net}(\lambda, \phi, s, t)
  =
  P_\mathrm{net,Mob}(\lambda, \phi, s, d(t)) \\
  + \frac{\partial P_\mathrm{net}}{\partial r}(\lambda, \phi, s, d(t))
  [r(\lambda, \phi, s, t) - r_\mathrm{Mob}(\lambda, \phi, s, d(t))] \\
  + \frac{\partial P_\mathrm{net}}{\partial T}(\lambda, \phi, s, d(t))
  [T(\lambda, \phi, s, t) - T_\mathrm{Mob}(\lambda, \phi, s, d(t))] \\
  + \frac{\partial P_\mathrm{net}}{\partial \Sigma}
  (\lambda, \phi, s, d(t))
  [\Sigma(\lambda, \phi, s, t) - \Sigma_\mathrm{Mob}
  (\lambda, \phi, s, d(t))] \\
  + R_\mathrm{het,corr}(\lambda, \phi, s, t) r(\lambda, \phi, s, t)
\end{multline*}
Ou encore :
\begin{displaymath}
  S = P_\mathrm{net,Mob} N + \ldots
  + R_\mathrm{het,corr} n_{\mathrm{O}_3}
\end{displaymath}
Donc $R_\mathrm{het,corr}$ est le taux de production (algébrique) par
chimie hétérogène par unité de volume et par molécule d'ozone.

Soit une position dans l'atmosphère. Soit $\pseudov{\ud^2 S} = \ud^2 S
\vec u_r$ une surface élémentaire à cette position. Notons $\ud^2
\tau$ le volume compris dans l'angle solide de sommet le centre de la
Terre, limitant $\ud^2 S$, et aux rayons $r'$ supérieurs au rayon $r$
de la position considérée.  J'interprète $\Sigma$ et
$\Sigma_\mathrm{Mob}$ comme les nombres de molécules d'ozone dans
$\ud^2 \tau$, divisés par $\ud^2 S$. D'où :
\begin{displaymath}
  \mu_{\mathrm{O}_3} \Sigma
  = \frac{1}{r^2} \int_r ^{+ \infty} q \rho r'^2 \ud r'
\end{displaymath}
Avec l'équilibre hydrostatique vertical :
\begin{displaymath}
  \mu_{\mathrm{O}_3} \Sigma
  = \frac{1}{r^2} \int_0 ^p q r'^2 \frac{\ud p'}{g}
\end{displaymath}

$a_2$, $a_4$ et $a_6$ sont constants sur les quatre niveaux de
pression les plus proches de la surface, sur toutes les latitudes et
tous les mois : $a_2 \approx - 2 \cdot 10^{-6}$, $a_4 \equiv 0$, $a_6
\equiv 0$. L'intervalle de pression correspond à une couche de 400 ou
500 m au dessus de la surface. Pour $a_2$, il y a une discontinuité
entre la valeur dans cette couche proche de la surface et les valeurs
au-dessus. Cf.
\href{file:///user/guez/Documents/Datasets/Ozone/Cariolle/Cariolle_v2_3/a2.ps}{exemple
  de variation de $a_2$}.

Ordres de grandeur dans \verb+coefoz_v2_3.nc+ :
\begin{displaymath}
  \begin{array}{rcl}
  - 10^{-11} \lesssim & P_\mathrm{net,Mob} & \lesssim 10^{-12} \\
  - 10^{-5} \lesssim & a_2 & \lesssim 10^{-11} \\
  10^{-8} \lesssim & r_\mathrm{Mob} & \lesssim 10^{-5} \\
  - 10^{-12} \lesssim & a_4 & \lesssim 10^{-13} \\
  184 \le & T_\mathrm{Mob} & \le 320 \\
  - 10^{-27} \lesssim & a_6 & \lesssim 10^{-28} \\
  10^{12} \lesssim & \Sigma_\mathrm{Mob} & \lesssim 10^{19} \\
  10^{-13} \lesssim & - R_\mathrm{het} & \lesssim 10^{-5}    
  \end{array}
\end{displaymath}
Deux valeurs de $|a_6|$ sont non nulles inférieures à \np{1.2e-38} :
environ \np{7e-39} et \np{2e-39}. \np{1.2e-38} est le plus petit
nombre strictement positif normalisé dans l'ensemble IEEE des réels
simple précision. Les deux petites valeurs de $|a_6|$ sont donc
dénormalisées dans le fichier NetCDF. La perte de précision n'est pas
importante. Ce problème ne se produit pas pour les autres
coefficients.

D'après D. Cariolle, on peut faire varier l'intensité de la
destruction hétérogène en changeant le coefficient A8 en fonction de
la charge en chlore total. Sinon le coefficient A3 est représentatif
des années 2000 et on suppose que pour changer d'année de référence,
il suffit de changer la destruction hétérogène (modulation en fonction
du chlore total, voir fichier readme). Le terme de variation en
température assure la répercussion des changements du CO2 via le
transfert radiatif.

\section[Fraction molaire à fraction massique d'ozone]{Passage
  de la fraction molaire à la fraction massique d'ozone}

Nous avons besoin de la masse moléculaire moyenne de l'air. Nous
pourrions considérer l'air comme un mélange de trois espèces : une
espèce fictive ``air sec'', l'eau (vapeur) et l'ozone. Nous supposons
connue la fraction massique d'eau (elle est calculée par
\verb+etat0+). Notons ici $q_i$ la fraction massique de l'espèce $i$
et $x_i$ sa fraction molaire. Nous avons donc le système de quatre
équations :
\begin{displaymath}
  \left\{
    \begin{array}{ll}
      q_{\mathrm{H}_2\mathrm{O}}
      & = \frac{\mu_{\mathrm{H}_2\mathrm{O}}}{\mu_\mathrm{moy}}
      x_{\mathrm{H}_2\mathrm{O}} \\
      \mu_\mathrm{moy}
      & = \mu_{\mathrm{H}_2\mathrm{O}} x_{\mathrm{H}_2\mathrm{O}}
      + \mu_{\mathrm{O}_3} x_{\mathrm{O}_3}
      + \mu_\textrm{air sec} x_\textrm{air sec} \\
      1 & = x_{\mathrm{H}_2\mathrm{O}} + x_{\mathrm{O}_3}
      + x_\textrm{air sec} \\
      q_{\mathrm{O}_3} & = \frac{\mu_{\mathrm{O}_3}}{\mu_\mathrm{moy}}
      x_{\mathrm{O}_3}
    \end{array}
  \right.
\end{displaymath}
pour les quatre inconnues : $\mu_\mathrm{moy}$,
$x_{\mathrm{H}_2\mathrm{O}}$, $x_\textrm{air sec}$, $q_{\mathrm{O}_3}$. La
solution de ce système est :
\begin{equation}
  \label{eq:qO3}
  \left\{
    \begin{array}{ll}
      \mu_\mathrm{moy}
      & =
      \frac{[(\mu_{\mathrm{O}_3} - \mu_\textrm{air sec}) x_{\mathrm{O}_3}
        + \mu_\textrm{air sec}] \mu_{\mathrm{H}_2\mathrm{O}}}
      {\mu_{\mathrm{H}_2\mathrm{O}}
        + (\mu_\textrm{air sec} - \mu_{\mathrm{H}_2\mathrm{O}})
        q_{\mathrm{H}_2\mathrm{O}}} \\
      q_{\mathrm{O}_3}
      & =
      \frac{\mu_{\mathrm{O}_3}
        [(\mu_\textrm{air sec} - \mu_{\mathrm{H}_2\mathrm{O}})
        q_{\mathrm{H}_2\mathrm{O}}
        + \mu_{\mathrm{H}_2\mathrm{O}}]}
      {\mu_{\mathrm{H}_2\mathrm{O}}
        \left[
          \mu_{\mathrm{O}_3}
          + \mu_\textrm{air sec}
          \left(
            \frac{1}{x_{\mathrm{O}_3}} - 1
          \right)
        \right]}
    \end{array}
  \right.
\end{equation}
Les valeurs typiques semblent être :
\begin{align*}
  \max q_{\mathrm{H}_2\mathrm{O}} \# 10^{-2} \\
  \max x_{\mathrm{O}_3} \# 10^{-5}
\end{align*}
L'écart entre $\mu_\mathrm{moy}$ et $\mu_\textrm{air sec}$ est alors
$\lesssim 10^{-2}$, et essentiellement dû à la présence de l'eau. Une
approximation intermédiaire entre :
\begin{displaymath}
  \left\{
    \begin{array}{ll}
      \mu_\mathrm{moy} & = \mu_\textrm{air sec} \\
      q_{\mathrm{O}_3} & = \frac{\mu_{\mathrm{O}_3}}{\mu_\textrm{air sec}}
      x_{\mathrm{O}_3}
    \end{array}
  \right.
\end{displaymath}
et le système (\ref{eq:qO3}) serait de calculer $\mu_\mathrm{moy}$ en
ne tenant compte que de la présence de l'eau :
\begin{displaymath}
  \mu_\mathrm{moy}
  =
  \frac{\mu_\textrm{air sec}}
  {1 +
    \left(
      \frac{\mu_\textrm{air sec}}{\mu_{\mathrm{H}_2\mathrm{O}}} - 1
    \right)
    q_{\mathrm{H}_2\mathrm{O}}}   
\end{displaymath}

\section{Choix du type de remaillage}

Copiant ce qui était déjà fait pour la vapeur d'eau, j'ai d'abord
choisi de remailler verticalement les coefficients de Cariolle en
interpolant une spline cubique de la pression. (Noter que ce n'est pas
une spline cubique du logarithme de la pression.) Si on interpole
simplement les données de Cariolle et que la pression à laquelle on
interpole est à l'extérieur du tableau de pressions de Cariolle alors
la formule d'interpolation devient sans crier gare une formule
d'extrapolation. C'est la formule pour l'intervalle extrême qui est
utilisée. Pour mieux contrôler cette extrapolation, j'ai ajouté aux
données de Cariolle un point correspondant à une pression nulle. Le
but était d'avoir approximativement une fraction molaire constante
au-dessus du point le plus haut de Cariolle et une production qui tend
vers 0.

Mais l'interpolation avec une spline cubique produit de fortes
oscillations. (Elle donne d'ailleurs souvent des fractions molaires
négatives.) Cf. exécutions
\href{file:///user/guez/Documents/Around_LMDZ/LMDZ_results.texfol/LMDZ_results.dvi}{28}
et
\href{file:///user/guez/Documents/Around_LMDZ/Around_LMDZE/Results_LMDZE_ce0l/check_coefoz_38.ps}{38}
de \verb+ce0l+. Le remaillage vertical est donc maintenant fait
par moyenne pour la plupart des coefficients chimiques, comme le
remaillage horizontal.

Considérons le remaillage de la fraction molaire $r_\mathrm{Mob}$ (le
raisonnement est le même pour $P_\mathrm{net,Mob}$). Nous avons choisi
un remaillage par moyenne donc nous voulons associer à une cellule de
la grille tri-dimensionnelle de LMDZ une moyenne de $r_\mathrm{Mob}$
sur cette cellule. Plus généralement, considérons un volume quelconque
de l'espace réel humain. Le plus sensé est d'associer à ce volume une
moyenne de $r_\mathrm{Mob}$ pondérée par la densité de l'air :
\begin{align*}
  \langle r_\mathrm{Mob} \rangle
  & = \frac{\iiint r_\mathrm{Mob} N \ud^3 \tau}{\iiint N \ud^3 \tau} \\
  & = \frac{\iiint (a + z)^2 r_\mathrm{Mob} N \cos\phi\,\ud z\, \ud \phi\, \ud \lambda}
  {\iiint (a + z)^2 N \cos\phi\, \ud z\, \ud \phi\, \ud \lambda}
\end{align*}
Opérons le
\hyperref{file:///user/guez/Documents/Commentaires_lectures/Autres_publications/1992-/Holton_2004/Chapter_3.texfol/chapter_3.dvi}{text}{z2p}{changement
  de variable} $(z, \phi, \lambda) \to (p, \phi, \lambda)$, avec
l'hypothèse d'équilibre hydrostatique vertical :
\begin{displaymath}
  \langle r_\mathrm{Mob} \rangle = \frac
  {
    \iiint (a + z)^2 r_\mathrm{Mob} \cos\phi \frac{\ud p}{\mu_\mathrm{moy} g}
    \ud \phi\, \ud \lambda
  }
  {
    \iiint (a + z)^2 \cos\phi \frac{\ud p}{\mu_\mathrm{moy} g}
    \ud \phi\, \ud \lambda
  }
\end{displaymath}
Si $z \ll a$ et $\mu_\mathrm{moy}$ et $g$ sont approximativement
constants sur le domaine considéré alors :
\begin{displaymath}
  \langle r_\mathrm{Mob} \rangle \approx \frac
  {\iiint r_\mathrm{Mob} \cos\phi\, \ud p\, \ud \phi\, \ud \lambda}
  {\iiint \cos\phi \, \ud p\, \ud \phi\, \ud \lambda}
\end{displaymath}
(Attention : même si $r_\mathrm{Mob}$ ne dépend pas de $\lambda$, l'intégrale sur
$\lambda$ ne se simplifie que si le domaine en $(p, \phi)$ est
indépendant de $\lambda$.) Considérons maintenant le cas particulier
où le domaine est une cellule de LMDZ. Nous considérons cette cellule
comme un parallélépipède rectangle dans l'espace $(p, \phi, \lambda)$,
d'où :
\begin{displaymath}
  \langle r_\mathrm{Mob} \rangle = \frac{1}{\Delta p \Delta(\sin\phi)}
  \iint r_\mathrm{Mob} \cos\phi\, \ud p\, \ud \phi
\end{displaymath}
Cf. figure (\ref{fig:Cellule}).
\begin{figure}[htbp]
  \centering
  \includegraphics{Cellule}
  \caption[Remaillage de la paramétrisation de la chimie de
  l'ozone]{En noir, cellule sur laquelle on moyenne les données
    concernant l'ozone. En rouge, cellules de Mobidic.}
  \label{fig:Cellule}
\end{figure}
Nous n'avons pas a priori $r_\mathrm{Mob}$ en tout point $(p, \phi)$,
nous avons un nombre fini de valeurs de $r_\mathrm{Mob}$. Il nous
reste donc à supposer une forme de variation de $r_\mathrm{Mob}$ en
fonction de $(p, \phi)$. Le plus simple est de supposer que
$r_\mathrm{Mob}$ est constant sur des cellules rectangulaires du plan
$(p, \phi)$. Nous devons choisir les limites de ces cellules. En
latitude, nous prenons les milieux des valeurs de Mobidic. En
pression, nous prenons les moyennes géométriques des valeurs de
Mobidic (si bien que les limites de cellule en pression sont à
mi-chemin des valeurs de Mobidic sur un axe logarithmique).

Pour les coefficients $a_2$ et $R_\mathrm{het}$, la démarche est moins
évidente. Supposons connu $a_2$ en tout point de ``l'espace humain'',
quelle valeur moyenne de $a_2$, $\overline{a_2}$, associer à un volume
$\tau$ ?  Considérons la paramétrisation simplifiée :
\begin{displaymath}
  P_\mathrm{net} = P_\mathrm{net,Mob} + a_2 (r - r_\mathrm{Mob})
\end{displaymath}
Nous voudrions que :
\begin{displaymath}
  \overline{P_\mathrm{net}}
  = \overline{P_\mathrm{net,Mob}}
  + \overline{a_2} (\bar r - \overline{r_\mathrm{Mob}})
\end{displaymath}
Ce qui donne :
\begin{displaymath}
  \overline{a_2}
  = \frac{\iiint a_2 (r - r_\mathrm{Mob}) N \ud^3 \tau}
  {\iiint (r - r_\mathrm{Mob})N \ud^3 \tau}
\end{displaymath}
On peut adopter :
\begin{displaymath}
  \overline{a_2} = \frac{\iiint a_2 N \ud^3 \tau}{\iiint N \ud^3 \tau}
\end{displaymath}
et garder ainsi une certaine compatibilité avec l'hypothèse selon
laquelle $r_\mathrm{Mob}$ est constant sur une cellule de
Mobidic. Idem pour $R_\mathrm{het}$.

Pour $a_4$ et $T_\mathrm{Mob}$, nous calculons aussi la moyenne
pondérée par $N$.

Pour $\Sigma_\mathrm{Mob}$, verticalement, une moyenne ne me
semblerait pas avoir de sens. Je choisis de faire une interpolation au
milieu de couche de LMDZ (pression \verb+pls+). Quelle interpolation ?
On a :
\begin{equation}
  \label{eq:sigma}
  \frac{\partial \Sigma_\mathrm{Mob}}{\partial z} = - r_\mathrm{Mob} N \approx - \frac{r_\mathrm{Mob} p}{k_B T}
\end{equation}
En remaillant $r_\mathrm{Mob}$ et $T$, nous avons supposé qu'ils
étaient constants sur des intervalles de pression centrés sur les
niveaux de pression de Mobidic. Cf. figure
(\ref{fig:Sigma_interp_vert}).
\begin{figure}[htbp]
  \centering
  \includegraphics{Sigma_interp_vert}
  \caption[Interpolation verticale de
  $\Sigma_\mathrm{Mob}$]{Interpolation verticale de
    $\Sigma_\mathrm{Mob}$. Les disques bleus sont les valeurs de
    $\Sigma_\mathrm{Mob}$ avant remaillage vertical. La ligne brisée
    noire (points anguleux noirs) est la fonction
    $\Sigma_\mathrm{Mob}(p)$ qu'il faudrait normalement supposer. La
    ligne brisée rouge (points anguleux rouges) est la fonction
    $\Sigma_\mathrm{Mob}(p)$ que nous supposons en fait.}
  \label{fig:Sigma_interp_vert}
\end{figure}
Considérons donc un tel intervalle, centré sur la pression $p_1$,
correspondant à une altitude $z_1$.  Supposons aussi
$\mu_\mathrm{moy}$ et $g$ constants. Sur l'intervalle, l'échelle $H$
de gradient de pression est constante :
\begin{displaymath}
  p(z) = p(z_1) \exp\left(- \frac{z - z_1}{H}\right)
\end{displaymath}
Donc, en intégrant (\ref{eq:sigma}) sur $z$ :
\begin{displaymath}
  \Sigma_\mathrm{Mob}(p) - \Sigma_\mathrm{Mob}(p_1) = \frac{r_\mathrm{Mob}}{\mu_\mathrm{moy} g} (p - p_1)
\end{displaymath}
$\Sigma_\mathrm{Mob}$ varie linéairement avec la pression. Nous
connaissons les valeurs de $\Sigma_\mathrm{Mob}$ aux niveaux de
Mobidic, que nous supposons être les centres des intervalles de
fraction molaire constante, et non les limites des intervalles. Nous
pourrions commencer par calculer $\Sigma_\mathrm{Mob}$ aux limites des
intervalles. Notons $p_1$ et $p_2$ deux niveaux de pression successifs
de Mobidic, $r_{\mathrm{Mob},1}$ et $r_{\mathrm{Mob},2}$ les fractions
molaires correspondantes, $p_l = \sqrt{p_1 p_2}$ la limite entre les
deux intervalles de centres $p_1$ et $p_2$. Posons :
\begin{displaymath}
  \alpha = \frac{r_{\mathrm{Mob},1}}{r_{\mathrm{Mob},2}} \frac{p_l - p_1}{p_2 - p_l}
\end{displaymath}
On a :
\begin{align*}
  & \Sigma_\mathrm{Mob}(p_l) - \Sigma_{\mathrm{Mob},1}
  = \frac{r_{\mathrm{Mob},1}}{\mu_\mathrm{moy} g} (p_l - p_1) \\
  & \Sigma_{\mathrm{Mob},2} - \Sigma_\mathrm{Mob}(p_l)
  = \frac{r_{\mathrm{Mob},2}}{\mu_\mathrm{moy} g} (p_2 - p_l)
\end{align*}
donc :
\begin{displaymath}
  \Sigma_\mathrm{Mob}(p_l) = \frac{\Sigma_{\mathrm{Mob},1} + \alpha \Sigma_{\mathrm{Mob},2}}{1 + \alpha}
\end{displaymath}
Plus simplement, je choisis d'interpoler linéairement en pression
entre les valeurs de Mobidic.

Pour $\Sigma_\mathrm{Mob}$ horizontalement, nous moyennons simplement pour
conserver le nombre de molécules d'ozone au-dessus d'une surface :
\begin{align*}
  \bar \Sigma_\mathrm{Mob} & = \frac{\iint \Sigma_\mathrm{Mob} \ud^2 S}{S} \\
  & =
  \frac{\iint \Sigma_\mathrm{Mob} (a + z^2) \cos\phi \ud \phi \ud \lambda}
  {\iint (a + z^2) \cos\phi \ud \phi \ud \lambda} \\
  & \approx \frac{\iint \Sigma_\mathrm{Mob} \cos\phi \ud \phi \ud \lambda}
  {\iint \cos\phi \ud \phi \ud \lambda}
\end{align*}

Pour $a_6$, une moyenne pondérée par $N$ semble censée. $a_6(p) (\Sigma(p) -
\Sigma_\mathrm{Mob}(p))$ est une contribution à
$P_\mathrm{net}(p)$. LMDZ calcule la moyenne
$\overline{P_\mathrm{net}}$ pondérée par $N$ sur une cellule. Une
contribution à cette moyenne est :
\begin{displaymath}
  \frac{\int a_6 N (\Sigma - \Sigma_\mathrm{Mob}) \ud \tau}{\int N \ud \tau}
\end{displaymath}
$\overline{a_6} (\Sigma - \Sigma_\mathrm{Mob})$, avec $\Sigma$ la
valeur en milieu de couche dans LMDZ et $\Sigma_\mathrm{Mob}$ la
valeur interpolée, se rapproche de cette contribution.

Cf. figures (\ref{fig:regr_latit}), (\ref{fig:regr_p_stepav}) et
(\ref{fig:regr_p_lint}).
\begin{figure}[htbp]
  \centering
  \includegraphics{regr_latit}
  \caption[Remaillage en latitude pour l'ozone]{Remaillage en
    latitude. En bleu, les valeurs d'origine, en rouge les valeurs
    utilisées.}
  \label{fig:regr_latit}
\end{figure}
\begin{figure}[htbp]
  \centering
  \includegraphics{regr_p_stepav}  
  \caption[Remaillage en pression conservatif pour l'ozone]
  {Remaillage en pression par moyenne de fonction en escalier.  En
    bleu, les valeurs d'origine, en rouge les valeurs utilisées. La
    position de \texttt{p3d(i, j, 1)} par rapport à
    \texttt{plev(n\_plev)} est quelconque.}
  \label{fig:regr_p_stepav}
\end{figure}
\begin{figure}[htbp]
  \centering
  \includegraphics{regr_p_lint}  
  \caption[Interpolation linéaire en pression pour l'ozone]{Remaillage
    en pression par interpolation linéaire. Les calculs aux valeurs
    élevées de \texttt{pls} (premiers indices de niveau vertical)
    peuvent être des extrapolations. La valeur supplémentaire à
    pression nulle du maillage source contrôle l'extrapolation aux
    basses pressions.}
  \label{fig:regr_p_lint}
\end{figure}

\section[Remaillage : programme, moment,
entrées-sorties]{Remaillage : dans quel programme, à quel moment ?
  Quelles entrées-sorties ?}

Les coefficients chimiques doivent être remaillés en latitude pour LMDZ.
Ce remaillage en latitude peut être fait une fois pour toutes, dans
\verb+ce0l+. Par ailleurs, au cours d'une simulation, la pression
à une longitude, une latitude et un niveau vertical donnés change
parce que la pression à la surface change. Donc il faut en principe
faire le remaillage en pression des données de Cariolle à chaque pas
de temps physique. Enfin, nous voulons lisser les variations
temporelles des coefficients chimiques en les interpolant
journalièrement. Ce remaillage temporel peut être fait une fois pour
toutes, dans \verb+ce0l+.

En pratique, le remaillage en pression à chaque pas de temps n'est pas
indispensable parce que l'essentiel de la chimie de l'ozone, avec
notre paramétrisation, se déroule à haute altitude, où les niveaux
verticaux de LMDZ sont pratiquement des niveaux de pression fixés.
Nous pouvons donc nous demander si, pour simplifier le programme, nous
pouvons faire les trois remaillages (latitude, pression, temps) dans
\verb+ce0l+. Nous négligerions alors pour le remaillage vertical
l'évolution des pressions aux interfaces des couches. Mais la
\href{file:///user/guez/Documents/Travail autour de
  LMDZ/taille.ods}{taille du fichier contenant les coefficients
  remaillés} serait alors trop grande.\footnote{Le remaillage en
  pression dans \texttt{etat0\_lim} serait une bonne solution si nous
  utilisions les coefficients mensuels sans interpolation
  journalière.}

Nous devons donc faire un des trois remaillages dans \verb+gcm+.
Lequel ? On ne peut faire le remaillage vertical sans avoir fait le
remaillage en latitude. Le choix dans \verb+gcm+ se restreint donc aux
remaillages vertical et temporel. Pour le calcul de la production
chimique d'ozone, nous avons besoin à chaque pas de temps de cinq
coefficients tri-dimensionnels (taille totale : $(\mathtt{iim} + 1)
\times (\mathtt{jjm} + 1) \times \mathtt{llm} \times 5$). Si nous
partons des coefficients remaillés en pression dans \verb+ce0l+,
nous devons calculer ces cinq coefficients par interpolation
temporelle pour chaque jour. Si nous partons des coefficients
interpolés journalièrement dans \verb+ce0l+, nous pouvons aussi
nous contenter de calculer ces cinq coefficients par remaillage
vertical pour chaque jour. Les deux transformations :
\begin{displaymath}
  \xymatrix{
    (\mathtt{jjm} + 1) \times 45 \ar[d]_{\textrm{remaillage vertical}}
    & (\mathtt{iim} + 1) \times (\mathtt{jjm} + 1) \times \mathtt{llm}
    \times 12 \ar[ld]^{\textrm{remaillage temporel}} \\
    (\mathtt{iim} + 1) \times (\mathtt{jjm} + 1) \times \mathtt{llm}
  }
\end{displaymath}
demandent des nombres d'opérations du même ordre de grandeur :
$(\mathtt{iim} + 1) \times (\mathtt{jjm} + 1) \times \mathtt{llm}$. Le
plus sensé est donc de faire le remaillage en pression dans
\verb+gcm+. Pour que l'arrêt et le redémarrage du programme \verb+gcm+
soient sans conséquence, nous ne pouvons faire le remaillage en
pression moins d'une fois par jour. (En particulier, nous ne pouvons
le faire une fois par exéuction de \verb+gcm+, au début de \verb+gcm+,
à partir de la valeur courante du champ de pression.) Nous faisons
donc le remaillage vertical pour chaque jour, négligeant l'évolution
des pressions aux interfaces des couches pendant un jour.

Nous devons encore nous demander s'il vaut mieux lire tout le contenu
de \verb+coefoz_LMDZ.nc+ au début de l'exécution de \verb+gcm+ ou lire
dans le fichier une fois par jour les coefficients au bon jour. Si
nous lisons tout en une seule fois, nous gagnons en temps d'exécution
(nous limitons les accès au disque) mais nous augmentons la
\href{file:///user/guez/Documents/Around_LMDZ/taille.ods}{mémoire
  principale consommée}. La lecture une fois par jour simulé est-elle
trop fréquente ? On peut se référer à une durée de 40 s de temps
écoulé par jour simulé en résolution $96 \times 72 \times 50$, sur
deux processus MPI. Pour le moment, la programmation de la lecture une
fois par jour me paraît plus simple.

\verb+ce0l+ fait le remaillage en latitude et en temps des huit
coefficients et le remaillage vertical du coefficient $r$ seulement,
au jour initial seulement, pour servir d'état initial de l'abondance
d'ozone. Nous ne pouvons donc pas faire le remaillage en latitude et
temps après \verb+etat0+. \verb+regr_lat_time_coefoz+ a besoin de
\verb+rlatu+ et \verb+rlatv+, définis par \verb+inigeom+, appelé par
\verb+etat0+. Nous ne pouvons donc pas faire non plus le remaillage en
latitude et temps avant \verb+etat0+.

Dans quel fichier \verb+ce0l+ doit-il écrire les coefficients
remaillés ? Le fichier \verb+limit.nc+ n'est pas très adapté car les
coefficients ne sont pas sur la grille de la ``physique''. De plus,
les coefficients chimiques ne sont pas des conditions à la limite. Par
ailleurs, les coefficients chimiques sont les mêmes dans plusieurs
exécutions successives de \verb+gcm+, donc ils ne peuvent pas être
dans \verb+start+ ou \verb+startphy+. Nous créons donc un nouveau
fichier \verb+coefoz_LMDZ.nc+. Cf.
\href{file:///user/guez/Documents/Informatique_fonctionnement/Programs/Guide_IPSL_climate_models/inout
  LMDZE.odg}{schéma des entrées-sorties de \texttt{etat0\_lim} et
  \texttt{gcm}}. \verb+coefoz_LMDZ.nc+ contient donc les données de
Cariolle remaillées en latitude pour LMDZ et interpolées
journalièrement.

Principe de la procédure \verb+regr_lat_time_coefoz+ :
\begin{algorithmic}
  \FOR{chaque coefficient à remailler}
  \STATE remaillage en latitude
  \STATE remaillage en temps
  \ENDFOR
\end{algorithmic}
Cf. figure (\ref{fig:regr_lat_time_coefoz}).
\begin{figure}[htbp]
  \centering
  \includegraphics{regr_lat_time_coefoz}
  \caption[Schéma des entrées-sorties de
  \texttt{regr\_lat\_time\_coefoz}]{Schéma des entrées-sorties de
    \texttt{regr\_lat\_time\_coefoz}. \texttt{rlatv} donne les limites
    en latitude des cellules de LMDZ. Cf. figure (\ref{fig:Cellule}).}
  \label{fig:regr_lat_time_coefoz}
\end{figure}

\section[Évolution de la distribution d'ozone dans
\texttt{phytrac}]{Calcul de l'évolution de la distribution d'ozone
  dans \texttt{phytrac}}

D'après l'équation (\ref{eq:dqdt}) :
\begin{displaymath}
  \partial_t q
  = - \vec U \cdot \vec\nabla q
  + P_\mathrm{net} \frac{\mu_{\mathrm{O}_3}}{\mu_\mathrm{moy}}
\end{displaymath}
Notons $\vec r$ le vecteur position. \verb+phytrac+ calcule :
\begin{displaymath}
  q(\vec r, t + \mathtt{pdtphys}) = q(\vec r, t) + \textrm{termes de transport}
  + \int_t ^{t + \mathtt{pdtphys}} P_\mathrm{net}(\vec r, t')
  \frac{\mu_{\mathrm{O}_3}}{\mu_\mathrm{moy}} \ud t'
\end{displaymath}
Le dernier terme est la contribution de la chime de l'ozone. Nous
avons besoin de $P_\mathrm{net}$ à chaque point de la grille
horizontale de LMDZ et à chaque niveau vertical de LMDZ. Le fichier
\verb+coefoz_LMDZ.nc+ contenant les grandeurs $P_\mathrm{net,Mob}$,
$a_2$, $r_\mathrm{Mob}$, $a_4$, $T _\mathrm{Mob}$, $a_6$,
$\Sigma_\mathrm{Mob}$ et $R_\mathrm{het}$, \verb+phytrac+ doit
intégrer :
\begin{multline*}
  P_\mathrm{net} \frac{\mu_{\mathrm{O}_3}}{\mu_\mathrm{moy}}
  = P_\mathrm{net,Mob} \frac{\mu_{\mathrm{O}_3}}{\mu_\mathrm{moy}}
  + a_2 \left(q - r_\mathrm{Mob} \frac{\mu_{\mathrm{O}_3}}{\mu_\mathrm{moy}}\right)
  + a_4 \frac{\mu_{\mathrm{O}_3}}{\mu_\mathrm{moy}}
  (T - T_\mathrm{Mob}) \\
  + \frac{a_6}{\mu_\mathrm{moy}}
  \left(
    \frac{1}{r^2} \int_r ^{+ \infty} q \rho r'^2 \ud r'
    - \mu_{\mathrm{O}_3} \Sigma_\mathrm{Mob}
  \right)
  + R_\mathrm{het,corr} q
\end{multline*}

Dans \verb+phytrac+, nous n'avons accès qu'à la température et à la
masse volumique à l'instant courant, et à aucun autre instant.
Calculons donc l'intégrale comme si $T$ et $\rho$ étaient des
constantes pendant \verb+pdtphys+. La direction du Soleil, qui
intervient dans le terme de chimie hétérogène, pourrait être
considérée comme une fonction du temps. Je choisis pour simplifier de
considérer aussi la direction du Soleil comme une constante pendant un
pas de temps de la physique. J'utilise la procédure \verb+zenang+ pour
calculer la moyenne de la direction du Soleil pendant un pas de temps
de la physique. Frédéric H. pense que les résultats seront meilleurs que
si nous utilisions une direction instantanée. Récrivons alors
$P_\mathrm{net} \mu_{\mathrm{O}_3} / \mu_\mathrm{moy}$ en mettant en
évidence les constantes. Posons :
\begin{align*}
  & c := [P_\mathrm{net,Mob} - a_2 r_\mathrm{Mob}
  + a_4 (T - T_\mathrm{Mob}) - a_6 \Sigma_\mathrm{Mob}]
  \frac{\mu_{\mathrm{O}_3}}{\mu_\mathrm{moy}} \\
  & b := a_2 + R_\mathrm{het,corr} \\
  & a_{6m} := \frac{a_6}{\mu_\mathrm{moy}}
\end{align*}
$c$, $b$ et $a_{6m}$ sont des champs constants connus pendant
\verb+pdtphys+. On a :
\begin{equation}
  \label{eq:dq_dt}
  P_\mathrm{net} \frac{\mu_{\mathrm{O}_3}}{\mu_\mathrm{moy}}
  = c + b q + a_{6m} \frac{1}{r^2} \int_r ^{+ \infty} q \rho r'^2 \ud r'
\end{equation}
Si on calcule l'évolution de $q$ en ne tenant compte
que de la chimie (c'est-à-dire sans tenir compte des termes de
transport) alors on obtient :
\begin{displaymath}
  \partial_t q = P_\mathrm{net} \frac{\mu_{\mathrm{O}_3}}{\mu_\mathrm{moy}}
\end{displaymath}

Limitons-nous d'abord aux deux premiers termes dans le membre de
droite de (\ref{eq:dq_dt}) :
\begin{equation}
  \label{eq:dq_dt_c_a2}
  \partial_t q = c + b q
\end{equation}
La solution exacte de cette équation différentielle sur $q$ donne :
\begin{equation}
  \label{eq:delta_q_exact}
  \begin{array}{|ll}
    \delta q = (c + b q) \frac{e^{b \delta t} - 1}{b}
    & \textrm{si } b \ne 0 \\
    \delta q = c \delta t & \textrm{si } b = 0
  \end{array}
\end{equation}
pour un intervalle de temps $\delta t$ quelconque. \`A l'ordre 2 en
$b \delta t$, nous avons donc, que $b$ soit nul ou non :
\begin{equation}
  \label{eq:delta_q}
  \delta q = (c + b q) (1 + b \delta t / 2) \delta t
\end{equation}
(Ce qui revient à avoir intégré l'équation différentielle par la
méthode du point milieu. Cf. Press et al. [1992 318, équation
(16.1.2)].) Pour $\delta t = 1800$ s, on a :
\begin{displaymath}
  4 \cdot 10^{-11} \le |a_2| \delta t \le \np{0,02}
\end{displaymath}
Entre la valeur calculée de $e^x -1$ et la valeur calculée de $x + x^2
/ 2$, laquelle est la plus proche de la valeur exacte de $e^x -1$ ?
Pour $x \ge 10^{-2}$, c'est la valeur calculée de $e^x -1$. Pour $0
\le x \le 10^{-3}$, c'est la valeur calculée de $x + x^2 / 2$. En
conséquence, l'utilisation de (\ref{eq:delta_q}) à la place de
(\ref{eq:delta_q_exact}) est un bon choix.

Si on tient maintenant compte du terme de densité-colonne dans
(\ref{eq:dq_dt}), nous avons une équation intégro-différentielle aux
dérivées partielles. Nous commençons par une discrétisation spatiale
verticale, à une position horizontale fixée. Nous avons \verb+llm+
couches d'atmosphère et nous supposons que $q$ est uniforme dans
chaque couche. La fonction inconnue $q(\vec r, t)$ devient alors un
vecteur de fonctions du temps : $[q_1(t), \dots, q_\mathtt{llm}(t)]$.
La fonction connue $c(\vec r)$ devient un vecteur de constantes
connues : $[c_1, \dots, c_\mathtt{llm}]$, idem pour $b$ et $a_{6m}$.
Notons $r_k$ la position de la limite entre les couches $k-1$
et $k$. Pour chaque couche $k$, nous avons :
\begin{displaymath}
  \frac{\ud q_k}{\ud t}
  = c_k + b_k q_k
  + a_{6m,k} \frac{1}{r_k^2}
  \sum_{k'=k} ^{\mathtt{llm}} q_{k'} \int_{r_{k'}} ^{r_{k'+1}} \rho r^2 \ud r
\end{displaymath}
Supposons que la différence entre $r_\mathtt{llm}$  et
le rayon de la Terre soit négligeable. Alors pour $k \le k' \le \mathtt{llm}
- 1$ :
\begin{equation}
  \label{eq:approx_column}
  \frac{1}{r_k^2} \int_{r_{k'}} ^{r_{k'+1}}\rho r'^2 \ud r'
  \approx \int_{z_{k'}} ^{z_{k'+1}} \rho \ud z
\end{equation}
Mais le cas de la couche supérieure est plus problématique :
$r_{\mathtt{llm}+1} = + \infty$. Nous pouvons faire l'approximation
(\ref{eq:approx_column}) pour $k' = \mathtt{llm}$ si $\rho$ décroît
suffisamment rapidement au dessus de $r_\mathtt{llm}$. C'est
l'approximation qui est faite dans \verb+phytrac+ pour le calcul de
\verb+zmasse+, et que je garde. Alors :
\begin{equation}
  \label{eq:dqk_dt}
  \frac{\ud q_k}{\ud t}
  = c_k + b_k q_k
  + a_{6m,k} \sum_{k'=k} ^{\mathtt{llm}} q_{k'} \mathtt{zmasse}_{k'}
\end{equation}
Nous nous sommes ramenés à un système d'équations différentielles du
premier ordre.

La somme partielle dans (\ref{eq:dqk_dt}) est la densité-colonne
d'ozone au dessus de la base de la couche $k$. Nous pourrions raffiner
en calculant à partir de là des densités-colonnes au dessus des
milieux de couches (par interpolation linéaire en pression) (d'autant
plus que $\Sigma_\mathrm{Mob}$ est interpolé en milieu de couche). Je
reste pour l'instant à l'équation (\ref{eq:dqk_dt}).

Les équations différentielles du premier ordre du système
(\ref{eq:dqk_dt}) sont linéaires à coefficients constants. Nous
pourrions chercher une solution exacte. Considérant l'étude que j'ai
faite de l'équation (\ref{eq:dq_dt_c_a2}), je choisis d'appliquer
simplement la méthode du point milieu.

$c$ et $b$ doivent être recalculés à chaque pas de temps de la
physique mais, parmi les termes composant $c$, les termes provenant de
Mobidic ne varient que pour le remaillage en pression: leur
combinaison peut être calculée une fois par jour. Distinguons donc :
\begin{displaymath}
  c_\mathrm{Mob}
  :=
  (P_\mathrm{net,Mob} - a_2 r_\mathrm{Mob}
  - a_6 \Sigma_\mathrm{Mob} - a_4 T_\mathrm{Mob})
  \frac{\mu_{\mathrm{O}_3}}{\mu_\mathrm{moy}}
\end{displaymath}
$c_\mathrm{Mob}$ regroupe les quatre coefficients ``climatologiques''.
On a :
\begin{displaymath}
  c = c_\mathrm{Mob} + a_4 \frac{\mu_{\mathrm{O}_3}}{\mu_\mathrm{moy}} T
\end{displaymath}
De même, pour le calcul de $b$, le filtrage de la chimie hétérogène
selon la latitude et la prise en compte du coefficient Clx peuvent
être faits une fois par jour. En résumé, nous avons distingué :
\begin{itemize}
\item la variable $q$ à intégrer pendant un pas de temps de la
  physique, qui est donc considérée comme dépendant du temps pendant
  \verb+pdtphys+ ;
\item les variables considérées comme constantes pendant
  \verb+pdtphys+ mais qui changent d'un pas de temps de la physique à
  l'autre ;
\item les variables qui ne changent qu'une fois par jour.
\end{itemize}

\section[\texttt{regr\_pr\_comb\_coefoz} et
\texttt{regr\_pr\_(av|int)\_coefoz}]{Les procédures
  \texttt{regr\_pr\_comb\_coefoz} et
  \texttt{regr\_pr\_(av|int)\_coefoz}}

Le remaillage en pression (cf. figure (\ref{fig:regridding_proced}))
fournit d'abord des variables ayant un double indice horizontal alors
que \verb+tr_seri+ dans \verb+phytrac+ a un indice horizontal simple.
\begin{figure}[htbp]
  \centering
  \includegraphics[width=\textwidth]{regridding_proced}
  \caption[Arbre des appels des procédures de remaillage]{Arbre des
    appels des procédures de remaillage. Les procédures du plus bas
    niveau sont celles qui effectuent réellement le remaillage. Elles
    sont les plus générales et ne font pas référence aux notions de
    pression, latitude ou temps. Entre parenthèses : les opérations
    non effectuées par les procédures de niveau inférieur.}
  \label{fig:regridding_proced}
\end{figure}
Il faut lire dans le fichier \verb+coefoz_LMDZ.nc+, remailler en
pression et faire la transformation d'indiçage une seule fois par jour
de simulation. De même, $c_\mathrm{Mob}$, $a_4 \mu_{\mathrm{O}_3} /
\mu_\mathrm{moy}$ et $a_{6m}$ doivent être calculés une seule fois par
jour. Toutes ces opérations sont faites dans la procédure
\verb+regr_pr_comb_coefoz+.  Quand pouvons-nous appeler
\verb+regr_pr_comb_coefoz+ ? \verb+calfis+ appelle forcément une seule
fois \verb+physiq+, qui appelle forcément une seule fois
\verb+phytrac+. Nous pourrions donc faire l'appel dans n'importe
laquelle de ces trois procédures. Nous choisissons \verb+phytrac+ pour
une plus grande modularité.

La procédure \verb+regr_pr_comb_coefoz+ lit en entier les huit
coefficients chimiques au jour courant et met à jour les cinq
variables de module : \verb+a2+, \verb+a4_mass+, \verb+c_mob+,
\verb+a6_mass+, \verb+r_het_interm+.  Nous regroupons ici les
définitions des quatre dernières :
\begin{align*}
  & a_{4m} := a_4 \frac{\mu_{\mathrm{O}_3}}{\mu_\mathrm{moy}} \\
  & c_\mathrm{Mob}
  := (P_\mathrm{net,Mob} - a_2 r_\mathrm{Mob} - a_6 \Sigma_\mathrm{Mob})
  \frac{\mu_{\mathrm{O}_3}}{\mu_\mathrm{moy}} - a_{4m} T_\mathrm{Mob} \\
  & a_{6m} := \frac{a_6}{\mu_\mathrm{moy}} \\
  & R_\mathrm{het,interm} = R_\mathrm{het} 1_{\phi \notin [- 45^\circ, 45^\circ]} 
  \left(\frac{\mathrm{Clx}}{\np{3,8} \cdot 10^{-9}}\right)^2
\end{align*}
Chacune des cinq variables de module est de profil \verb+(klon, llm)+.

La procédure \verb+regr_pr_(av|int)_coefoz+ est appelée une fois par
jour, pour chaque coefficient chimimque. Cf. figure (\ref{fig:regr_pr_coefoz}).
\begin{figure}[htbp]
  \centering
  \includegraphics{regr_pr_coefoz}
  \caption[Entrées-sorties de la procédure
  \texttt{regr\_pr\_(av|int)\_coefoz}]{Entrées-sorties de la procédure
    \texttt{regr\_pr\_(av|int)\_coefoz}. En rouge : les données, en
    noir : les traitements.}
  \label{fig:regr_pr_coefoz}
\end{figure}

\section{Vue d'ensemble informatique}

Cf. figure (\ref{fig:ozone_flow}).
\begin{figure}[htbp]
  \centering
  \includegraphics{ozone_flow}
  \caption[Vue d'ensemble de ce qui concerne le traceur ozone]{Vue
    d'ensemble de ce qui concerne le traceur ozone dans les programmes
    entiers \texttt{etat0\_lim} et \texttt{gcm}. On fait apparaître
    les fichiers, les principales procédures et variables concernées.}
  \label{fig:ozone_flow}
\end{figure}
Les résultats du programme \verb+gcm+ pour l'ozone apparaissent dans
cinq fichiers (cf. figure (\ref{fig:ozone_flow})). Dans \verb+histrac.nc+,
ce sont les variables :
\begin{itemize}
\item \verb+float O3(time_counter, presnivs, lat, lon)+
\item \verb+flO3+ flux
\item \verb+d_tr_th_O3+ tendance thermique
\item \verb+d_tr_cv_O3+ tendance convection
\item \verb+d_tr_cl_O3+ tendance couche limite
\end{itemize}
La différence entre \verb+O3+ dans \verb+histrac.nc+ et :
\begin{verbatim}
float O3VL1(time_counter, sigss, lat, lon)
\end{verbatim}
dans \verb+dyn_hist_ave.nc+ n'est pas claire pour moi. Une longitude
en plus, $180^\circ$, dans \verb+dyn_hist_ave.nc+. Les latitudes sont
les mêmes. Cf.
\hyperref{file:/user/guez/Documents/Utilisation_LMDZ/Utilisation_LMDZ.texfol/Utilisation-LMDZ.dvi}{section}{ozone}{résultats}.
Faire varier les paramètres dans \verb+physiq.def+ :
\begin{verbatim}
OK_mensuel=y
OK_instan=y
lev_histhf=4
lev_histmth=2
\end{verbatim}
n'apporte aucune sortie supplémentaire sur l'ozone. Cf. exécutions 58
à 63 de \verb+gcm+.

Diverses procédures s'occupent de remaillage. Elles sont de
généralités différentes. Cf. figure (\ref{fig:regridding_proced}).
\begin{description}
\item[\texttt{regr\_pr\_comb\_coefoz}] : lecture, remaillage en pression,
  empaquetage, combinaison
\item[\texttt{regr\_pr\_coefoz}] : lecture, remaillage en pression, empaquetage
\item[\texttt{regr\_lat\_time\_coefoz}] : lecture, remaillage en temps et
  latitude, écriture
\item[\texttt{regr\_pr\_o3}] : lecture, remaillage en pression
\item[\texttt{regr1\_lint}, \texttt{regr1\_step\_av},
  \texttt{regr3\_lint}] : remaillage
\end{description}

\end{document}
